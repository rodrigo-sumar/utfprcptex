%% Exemplo de Utilizacao do Estilo de formatacao utfprcptex  (http://http://code.google.com/p/utfprcptex/)
%% para elabora��o de Teses, Disserta��es, etc...
%% Autores: Rodrigo Rodrigues Sumar (sumar@utfpr.edu.br)
%%          Bruno Augusto Ang�lico (bangelico@utfpr.edu.br)
%% Colaboradores:
%%
%%


\documentclass[oneside]{utfprcptex}
%\input{macro}
\usepackage[pdftex,colorlinks=false,plainpages=false,pdfpagelabels]{hyperref}
%% A seguinte linha evita algumas mensagens de warning por parte do hyperref.
%% Comentar se o pacote hyperref n�o estiver sendo utilizado.
\pdfstringdefDisableCommands{\edef\uppercase{}}
\pdfstringdefDisableCommands{%
\let\MakeUppercase\relax
}
%configuracao correta das referencias bibliogr�ficas.
\usepackage[alf,abnt-emphasize=bf,bibjustif,recuo=0cm, abnt-etal-cite=3, abnt-etal-list=0]{abntcite}


\usepackage{amsmath,amsfonts,amssymb} % pacote matematico
\usepackage{graphicx} % pacote grafico
% Para fonte Times (Nimbus Roman) descomente linha abaixo
\fontetipo{times}
\setlength\fboxrule{1.5pt}
% Para fonte Helvetica (Nimbus Sans) descomente linhas abaixo
%\fontetipo{arial}


% ---------- Preambulo ----------
\instituicao{Universidade Tecnol\'ogica Federal do Paran\'a}
\unidade{C�mpus Corn�lio Proc�pio}
\diretoria{Diretoria de Gradua��o}
\curso{Curso Superior de T�cnologia em Automa��o Industrial}
\documento{Proposta para Trabalho de Conclus�o}
\titulacao{T�cnologo}
\area{Automa��o Industrial}
\titulo{\MakeUppercase{T\'itulo em Portugu\^es}} % titulo do trabalho em portugues
\autor{Nome do Autor} % autor do trabalho
\codigoaluno{00955390}
\emailaluno{aluno@email.com}
\fonealuno{(XX)XXXX-XXXX}
\orientador[ORIENTADOR]{Prof. Dr.}{Nome do Orientador} % nome do orientador do trabalho
\emailorientador{orientador@email.com}
\coorientador[ORIENTADORA]{Profa. Dra.}{Nome da Co-orientadora} % <- no caso de co-orientadora, usar esta sintaxe
\emailcoorientador{coorientador@email.com}
\local{Corn�lio Proc�pio - PR}
\data{\the\year} % ano autom�tico
\comentario{\UTFPRdocumentodata\ apresentada ao \UTFPRcursodata\ da \ABNTinstituicaodata\ como requisito parcial para obten\c{c}\~ao do t�tulo de ``\UTFPRtitulacaodata\ em \UTFPRareadata''.}
%---------- Inicio do Documento ----------
\begin{document}
\capapre
\folhaderosto[semficha] % geracao automatica da folha de rosto
% sumario
\sumario % geracao automatica do sumario
\hypersetup{pageanchor=true}
%---------- Inicio do Texto ----------


% recomenda-se a escrita de cada capitulo em um arquivo texto separado (exemplo: intro.tex, fund.tex, exper.tex, concl.tex, etc.) e a posterior inclus�o dos mesmos no mestre do documento utilizando o comando \input{}, da seguinte forma:
\chapter{Introdu��o}
\label{cap:introducao}

%%%%%%%%%%%%%%%%%%%%

Nos �ltimos anos os...


Pesquisas indicam que...


Por mais de meio s�culo... \cite{Anderson1977}


\begin{figure}[!htb]\centering
    \includegraphics[width=7cm]{introducao/imagens/meioseculo.jpg}\\
  \caption{Meio S�culo}\label{meioseculo}
  \fonte{Google}
\end{figure}




\chapter{Problema de Pesquisa}

� poss�vel otimizar

\chapter{Justificativa} 

Atualmente o ambiente industrial....
\chapter{Objetivos}
\section{Objetivo Geral} 

O objetivo do trabalho esta vinculado com a.....
\section{Objetivos Espec�ficos}


\begin{description}
  \item[] Aplicar...
  \item[] Estudar...
  \item[] Desenvolver...
\end{description}  
\chapter{Metodologia} 

Ser� realizada, primeiramente, uma revis�o...

Consolidar os algoritmos que ser�o...

Estudo e modifica��o da planta...
\chapter{Fundamenta��o Te�rica} 


Por meio de suas...


Originalmente, os controladores...


%=======================================================================
\chapter {Cronograma de Trabalho}
\label{cap:cronograma}
%=======================================================================

Visando atingir os objetivos propostos apresenta-se um cronograma
de atividades a ser realizado no �mbito do Departamento de
Automa��o e Sistemas (DAS/UFSC). Estas atividades e o cronograma
est�o ilustrados nas tabelas \ref{tb:atividades} e
\ref{tb:cronograma}, respectivamente.

%%%% INICIO ATIVIDADES PREVISTAS %%%%%%%%%%%%%%%%%

\begin{table}[!htb]
  \centering
  \caption{Atividades Previstas}\label{tb:atividades}
  \begin{tabular}{cp{12cm}}
    \hline \hline &\\[-0.4cm]
    {\bf Atividades} & \multicolumn{1}{c}{\bf Descri��o} \\
    \hline
    &\\[-0.4cm]
    \textbf{A} & Revis�o bibliogr�fica. \\[0.2cm]
    \textbf{B} &  Estudo de novas representa��es.\\[0.2cm]
    \textbf{C} &  Aplica��o dos algoritmos.\\[0.2cm]
    \textbf{D} &  Desenvolvimento da interface. \\[0.2cm]
    \textbf{E} &  Valida��o dos resultados.\\[0.2cm]
    \textbf{F} &  Elabora��o da monogr�fia.\\[0.2cm]
    \textbf{G} &  Defesa.\\[0.2cm]
    \hline \hline
  \end{tabular}
\end{table}

%%%% FIM ATIVIDADES PREVISTAS %%%%%%%%%%%%%%%%%


%%%%%% INICIO DO CRONOGRAMA %%%%%%%%%%%%%%

\begin{table}[!htb]
  \centering \fontsize{8}{14}%\tiny
  \caption{Cronograma de Atividades}\label{tb:cronograma}
  \begin{tabular}{|c|c|c|c|c|c|c|c|c|c|c|}
    \hline
    {\normalsize\bf Ano} & \multicolumn{6}{c|}{\normalsize\bf 2004} &\multicolumn{4}{c|}{\normalsize\bf 2005}\\
    \hline
 {\normalsize\bf M�s} &
 \multirow{2}*{\bf Jul}&\multirow{2}*{\bf Ago}& \multirow{2}*{\bf Set}&\multirow{2}*{\bf Out}&\multirow{2}*{\bf Nov}&
 \multirow{2}*{\bf Dez}&\multirow{2}*{\bf Jan}&\multirow{2}*{\bf Fev}&  \multirow{2}*{\bf Mar}&\multirow{2}*{\bf Abr}\\
   \cline{1-1}
{\bf Atividade}    & & & & & & & & & & \\
\hline
{\normalsize\bf A} &\multicolumn{2}{c|}{\cellcolor{meucinza}}&     &   &    &    &       &         &      & \\
\hline
{\normalsize\bf B} &   &    &\multicolumn{2}{c|}{\cellcolor{meucinza}} &  &   &   &         &            &  \\
\hhline{>{\arrayrulecolor{black}}---->{\arrayrulecolor{meucinza}}->{\arrayrulecolor{black}}------}
{\normalsize\bf C} &   &    &    &\multicolumn{3}{c|}{\cellcolor{meucinza}}   &       &         &        &   \\
\hhline{>{\arrayrulecolor{black}}----->{\arrayrulecolor{meucinza}}-->{\arrayrulecolor{black}}----}
{\normalsize\bf D} &    &  &    &  &\multicolumn{2}{c|}{\cellcolor{meucinza}} &    &       &                        & \\
\hhline{>{\arrayrulecolor{black}}------>{\arrayrulecolor{meucinza}}->{\arrayrulecolor{black}}----}
{\normalsize\bf E} &     &      &        &        &       &\multicolumn{2}{c|}{\cellcolor{meucinza}} &       &        &    \\
\hhline{>{\arrayrulecolor{black}}------->{\arrayrulecolor{meucinza}}->{\arrayrulecolor{black}}---}
{\normalsize\bf F} &   &      &       &       &       &      &\multicolumn{3}{c|}{\cellcolor{meucinza}} &     \\
\hhline{>{\arrayrulecolor{black}}-------->{\arrayrulecolor{meucinza}}-->{\arrayrulecolor{black}}-}
{\normalsize\bf G} &     &   &    &      &       &    &       &\multicolumn{3}{c|}{\cellcolor{meucinza}}   \\
\hline
  \end{tabular}
\end{table}
%%%%%% FIM DO CRONOGRAMA %%%%%%%%%%%%%%


%=======================================================================


%---------- Referencias ----------
\bibliography{referencias/referencias} % geracao automatica das referencias a partir do arquivo reflatex.bib

\end{document} 