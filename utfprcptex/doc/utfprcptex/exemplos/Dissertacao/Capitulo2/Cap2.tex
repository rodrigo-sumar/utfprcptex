%---------- Segundo Capitulo ----------
\chapter{Desenvolvimento}
\label{chap:desenv}

A seguir ilustra-se a forma de incluir figuras, tabelas, equa\c{c}\~oes, siglas e s\'imbolos no documento, obtendo indexa\c{c}\~ao autom\'atica em suas respectivas listas. A numera\c{c}\~ao sequencial de figuras, tabelas e equa\c{c}\~oes ocorre de modo autom\'atico. Refer\^encias cruzadas s\~ao obtidas atrav\'es dos comandos {\ttfamily \textbackslash label\{\}} e {\ttfamily \textbackslash ref\{\}}. Por exemplo, n\~ao \'e necess\'ario saber que o n\'umero deste cap\'itulo \'e~\ref{chap:desenv} para colocar o seu n\'umero no texto. Isto facilita muito a inser\c{c}\~ao, remo\c{c}\~ao ou reloca\c{c}\~ao de elementos numerados no texto (fato corriqueiro na escrita e corre\c{c}\~ao de um documento acad\^emico) sem a necessidade de renumer\'a-los todos.

\section{Figuras}

Na figura~\ref{fig:dummy} \'e apresentado um exemplo de gr\'afico flutuante. Esta figura aparece automaticamente na lista de figuras. Para uso avan\c{c}ado de gr\'aficos no \LaTeX, recomenda-se a consulta de literatura especializada~\cite{Goossens2007}.


\begin{figure}[!htb]
	\centering
\fbox{\begin{minipage}{0.97\linewidth}\centering
	\includegraphics[width=0.2\textwidth]{./dummy.png} \end{minipage}}% <- formatos PNG, JPG e PDF
	\caption[Exemplo de uma figura]{Exemplo de uma figura onde aparece uma imagem sem nenhum significado especial.}
	\fonte{\cite{abnTeX2009}}
	\label{fig:dummy}
\end{figure}


\section{Quadros}

Tamb\'em \'e apresentado o exemplo do quadro~\ref{quad:descricao}, que aparece automaticamente na lista de quadros. Informa\c{c}\~oes sobre a constru\c{c}\~ao de quadros no \LaTeX\ podem ser encontradas na literatura especializada~\cite{Lamport1986,Buerger1989,Kopka2003,Mittelbach2004}.

\begin{table}[!htb]
\centering
	\begin{tabular}{|l|p{0.5\linewidth}|}
		\hline
		 \textbf{�reas de Desenvolvimento} & \textbf{Descri��o} \\
		\hline
		1. Compet�ncias sobre processos & Conhecimento nos processos de trabalho \\
        \hline
		2. Compet�ncias t�cnicas & Conhecimento t�cnico nas tarefas a serem desempenhadas e tecnologias
empregadas nestas tarefas \\
        \hline
		3. Compet�ncias sobre a organiza��o & Saber organizar os fluxos de trabalho \\
        \hline
		4. Compet�ncias de servi�o & Aliar as compet�ncias t�cnicas com o impacto que estas a��es ter�o para o
cliente consumidor \\
		\hline
        5. Compet�ncias sociais & Atitudes que sustentam o comportamento do indiv�duo: saber comunicar-se e
responsabilizar-se pelos seus atos. \\
		\hline
	\end{tabular}
    \quadro{�reas de Desenvolvimento de Compet�ncias}\label{quad:descricao}
	\fonte{Zarifian (1999) apud Fleury e Fleury (2004).}
\end{table}

%%%%%%%%%%%%%%%%%%%%
\section{Tabelas}

Tamb\'em \'e apresentado o exemplo da tabela~\ref{tab:correlacao}, que aparece automaticamente na lista de tabelas. Informa\c{c}\~oes sobre a constru\c{c}\~ao de tabelas no \LaTeX\ podem ser encontradas na literatura especializada~\cite{Lamport1986,Buerger1989,Kopka2003,Mittelbach2004}.\\\vspace{12pt}


\begin{table}[!htb]
	\centering
	\caption[Exemplo de uma tabela]{Exemplo de uma tabela mostrando a correla\c{c}\~ao entre x e y.}
	\label{tab:correlacao}
	\begin{tabular}{cc}
		\hline
		x & y \\
		\hline
		1 & 2 \\
		3 & 4 \\
		5 & 6 \\
		7 & 8 \\
		\hline
	\end{tabular}
    \vspace{0pt} %%%% Deve ser acrescentado para que haja espa�o entre o final da tabela e a fonte.
    \fonte{Autoria pr\'opria.}
\end{table}




\section{Equa\c{c}\~oes}

A transformada de Laplace \'e dada na equa\c{c}\~ao~(\ref{eq:laplace}), enquanto a equa\c{c}\~ao~(\ref{eq:dft}) apresenta a formula\c{c}\~ao da transformada discreta de Fourier bidimensional\footnote{Deve-se reparar na formata\c{c}\~ao esteticamente perfeita destas equa\c{c}\~oes!}.

\begin{equation}
X(s) = \int\limits_{t = -\infty}^{\infty} x(t) \, \text{e}^{-st} \, dt
\label{eq:laplace}
\end{equation}

\begin{equation}
F(u, v) = \sum_{m = 0}^{M - 1} \sum_{n = 0}^{N - 1} f(m, n) \exp \left[ -j 2 \pi \left( \frac{u m}{M} + \frac{v n}{N} \right) \right]
\label{eq:dft}
\end{equation}
%\clearpage


\section{Siglas e s\'imbolos}

O pacote \textsc{abn}\TeX\ permite ainda a defini\c{c}\~ao de siglas e s\'imbolos com indexa\c{c}\~ao autom\'atica atrav\'es dos comandos {\ttfamily \textbackslash sigla\{\}\{\}} e {\ttfamily \textbackslash simbolo\{\}\{\}}. Por exemplo, o significado das siglas \sigla{PPGEE}{Programa de Programa de P�s-Gradua��o em Engenharia El�trica}, \sigla{COELT}{Coordena��o de Eletrot�cnica} e \sigla{UTFPR}{Universidade Tecnol�gica Federal do Paran�} aparecem automaticamente na lista de siglas, bem como o significado dos s\'imbolos \simbolo{$\lambda$}{comprimento de onda},\simbolo{$v$}{velocidade} e \simbolo{$f$}{frequ\^encia} aparecem automaticamente na lista de s\'imbolos. Mais detalhes sobre o uso destes e outros comandos do \textsc{abn}\TeX\ s\~ao encontrados na sua documenta\c{c}\~ao espec\'ifica~\cite{abnTeX2009}.

O comando {\ttfamily \textbackslash abrevi\{\}\{\}} da classe \utfprtex permite a defini��o de abreviaturas. Por exemplo, o significado das abreviaturas para \abrevi{coef.}{Coeficiente},  \abrevi{V. Exa.}{Vossa Excel�ncia}, \abrevi{hab.}{Habitantes} aparecem automaticamente na lista de abreviaturas. 








%\begin{teorema}
%fsadfd
%\end{teorema}
%
%\begin{prova}
%dfasdfa
%\end{prova}
%
%\begin{lema}
%fgfgfg
%\end{lema}
%
%\begin{algorithm}
%\caption{Calculate $y = x^n$}
%\label{alg1}
%\begin{algorithmic}
%\REQUIRE $n \geq 0 \vee x \neq 0$
%\ENSURE $y = x^n$
%\STATE $y \leftarrow 1$
%\IF{$n < 0$}
%\STATE $X \leftarrow 1 / x$
%\STATE $N \leftarrow -n$
%\ELSE
%\STATE $X \leftarrow x$
%\STATE $N \leftarrow n$
%\ENDIF
%\WHILE{$N \neq 0$}
%\IF{$N$ is even}
%\STATE $X \leftarrow X \times X$
%\STATE $N \leftarrow N / 2$
%\ELSE[$N$ is odd]
%\STATE $y \leftarrow y \times X$
%\STATE $N \leftarrow N - 1$
%\ENDIF
%\ENDWHILE
%\end{algorithmic}
%\end{algorithm} 