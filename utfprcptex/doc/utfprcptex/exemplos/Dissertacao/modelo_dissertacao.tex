%% Exemplo de Utilizacao do Estilo de formatacao utfprcptex  (http://http://code.google.com/p/utfprcptex/)
%% para elabora��o de Teses, Disserta��es, etc...
%% Autores: Rodrigo Rodrigues Sumar (sumar@utfpr.edu.br)
%%          Bruno Augusto Ang�lico (bangelico@utfpr.edu.br)
%% Colaboradores:
%%
%%


\documentclass[openright]{utfprcptex} %openright = o capitulo comeca sempre em paginas impares
%\documentclass[oneside]{normas-utf-tex} %oneside = para dissertacoes com numero de paginas menor que 100 (apenas frente da folha)

\usepackage[pdftex,colorlinks=false,plainpages=false,pdfpagelabels]{hyperref}
%% A seguinte linha evita algumas mensagens de warning por parte do hyperref.
%% Comentar se o pacote hyperref n�o estiver sendo utilizado.
\pdfstringdefDisableCommands{\edef\uppercase{}}
\pdfstringdefDisableCommands{%
\let\MakeUppercase\relax
}

\usepackage[alf,abnt-emphasize=bf,bibjustif,recuo=0cm, abnt-etal-cite=3, abnt-etal-list=0]{abntcite} %configuracao correta das referencias bibliogr�ficas.

\usepackage[brazil]{babel} % pacote portugues brasileiro
\usepackage[latin1]{inputenc} % pacote para acentuacao direta
\usepackage{amsmath,amsfonts,amssymb} % pacote matematico
\usepackage{graphicx} % pacote grafico

% Para fonte Times (Nimbus Roman) descomente linha abaixo
\fontetipo{times}

% Para fonte Helvetica (Nimbus Sans) descomente linhas abaixo
%\fontetipo{arial}




%O comando a seguir diz ao Latex para salvar as siglas em um arquivo separado:
\fazlistasiglas[Lista de Abreviaturas e Siglas] % Par�metro opcional � o t�tulo da lista.

%Podem utilizar GEOMETRY{...} para realizar pequenos ajustes das margens. Onde, left=esquerda, right=direita, top=superior, bottom=inferior. P.ex.:
%\geometry{left=3.0cm,right=1.5cm,top=4cm,bottom=1cm}

% ---------- Preambulo ----------
\instituicao{Universidade Tecnol\'ogica Federal do Paran\'a}
\unidade{C�mpus Corn�lio Proc�pio}
\diretoria{Diretoria de Pesquisa e P�s-Gradua��o}
\coordenacao{Programa de P�s-Gradua��o em Engenharia El�trica}
\curso{Mestrado em Engenharia El�trica}
\documento{Disserta��o}
\nivel{Mestrado}
\titulacao{Mestre}
\area{Engenharia El�trica}
\titulo{\MakeUppercase{T\'itulo em Portugu\^es}} % titulo do trabalho em portugues
\title{\MakeUppercase{Title in English}} % titulo do trabalho em ingles
\autor{Nome do Autor} % autor do trabalho
\cita{SOBRENOME, Nome} % sobrenome (maiusculas), nome do autor do trabalho

\palavraschave{Palavra-chave 1, Palavra-chave 2, ...} % palavras-chave do trabalho
\keywords{Keyword 1, Keyword 2, ...} % palavras-chave do trabalho em ingles
\comentario{\UTFPRdocumentodata\ apresentada ao \UTFPRcoordenacaodata\ da \ABNTinstituicaodata\ como requisito parcial para obten\c{c}\~ao do t�tulo de ``\UTFPRtitulacaodata\ em \UTFPRareadata''.}



\orientador[Orientador]{Prof. Dr.}{Nome do Orientador} % nome do orientador do trabalho
%\orientador[Orientadora:]{Nome da Orientadora} % <- no caso de orientadora, usar esta sintaxe
%\coorientador{Nome do Co-orientador} % nome do co-orientador do trabalho, caso exista
\coorientador[Co-orientadora]{Profa. Dra.}{Nome da Co-orientadora} % <- no caso de co-orientadora, usar esta sintaxe
%\coorientador[Co-orientadores:]{Nome do Co-orientador} % no caso de 2 co-orientadores, usar esta sintaxe
%\coorientadorb{}	% este comando inclui o nome do 2o co-orientador

\coordenadorUTFPR[Coordenadora]{Grau}{Nome do coordenador} % Coordenador do PPGEEL
\local{Corn�lio Proc�pio} % cidade
\data{\the\year} % ano autom�tico

%%%% Folha de Aprova��o
\textoaprovacao{Esta \UTFPRdocumentodata\ foi julgada adequada para obten\c c\~ao do T\'itulo de ``\UTFPRtitulacaodata\ em \UTFPRareadata'' e aprovado em sua forma final pelo \UTFPRcoordenacaodata\ da \ABNTinstituicaodata.}
\primeiroassina[Doutor, UFSC]{Primeiro Membro}
\segundoassina{Segundo Membro}
\terceiroassina{Terceiro Membro}
\quartoassina{Quarto Membro}
\datadefesaUTFPR{29/07/2011}
%---------- Inicio do Documento ----------
\begin{document}
\hypersetup{pageanchor=false}
\capa % geracao automatica da capa
\folhaderosto % geracao automatica da folha de rosto
\termodeaprovacao % <- ainda a ser implementado corretamente

% dedicat�ria (opcional)
\begin{dedicatoria}
Texto da dedicat\'oria.
\end{dedicatoria}

% agradecimentos (opcional)
\begin{agradecimentos}
Texto dos agradecimentos.
\end{agradecimentos}

% epigrafe (opcional)
\begin{epigrafe}
Texto da ep\'igrafe.
\end{epigrafe}

%resumo
\begin{resumo}
Texto do resumo (m\'aximo de 500 palavras).
\end{resumo}

%abstract
\begin{abstract}
Abstract text (maximum of 500 words).
\end{abstract}


% listas (opcionais, mas recomenda-se a partir de 5 elementos)
\listadefiguras % geracao automatica da lista de figuras
\listadetabelas % geracao automatica da lista de tabelas
\listadesiglas % geracao automatica da lista de siglas
\listadesimbolos % geracao automatica da lista de simbolos

% sumario
\sumario % geracao automatica do sumario
\hypersetup{pageanchor=true}

%---------- Inicio do Texto ----------
% recomenda-se a escrita de cada capitulo em um arquivo texto separado (exemplo: intro.tex, fund.tex, exper.tex, concl.tex, etc.) e a posterior inclusao dos mesmos no mestre do documento utilizando o comando \input{}, da seguinte forma:
\input{capitulo1/cap1.tex}
\input{capitulo2/cap2.tex}
\input{capitulo3/cap3.tex}

%---------- Referencias ----------
\bibliography{referencias/referencias} % geracao automatica das referencias a partir do arquivo reflatex.bib


%---------- Apendices (opcionais) ----------
\apendice
\chapter{Nome do Ap\^endice}

Use o comando {\ttfamily \textbackslash apendice} e depois comandos {\ttfamily \textbackslash chapter\{\}}
para gerar t\'itulos de ap\^en-dices.


% ---------- Anexos (opcionais) ----------
\anexo
\chapter{Nome do Anexo}

Use o comando {\ttfamily \textbackslash anexo} e depois comandos {\ttfamily \textbackslash chapter\{\}}
para gerar t\'itulos de anexos.


% --------- Lista de siglas --------
%\textbf{* Observa\c{c}\~oes:} a lista de siglas nao realiza a ordenacao das siglas em ordem alfabetica
% Em breve isso sera implementado, enquanto isso:
%\textbf{Sugest\~ao:} crie outro arquivo .tex para siglas e utilize o comando \sigla{sigla}{descri\c{c}\~ao}.
%Para incluir este arquivo no final do arquivo, utilize o comando \input{}.
%Assim, Todas as siglas serao geradas na ultima pagina. Entao, devera excluir a ultima pagina da versao final do arquivo
% PDF do seu documento.


%-------- Citacoes ---------
% - Utilize o comando \citeonline{...} para citacoes com o seguinte formato: Autor et al. (2011).
% Este tipo de formato eh utilizado no comeco do paragrafo. P.ex.: \citeonline{autor2011}

% - Utilize o comando \cite{...} para citacoeses no meio ou final do paragrafo. P.ex.: \cite{autor2011}



%-------- Titulos com nomes cientificos (titulo, capitulos e secoes) ----------
% Regra para escrita de nomes cientificos:
% Os nomes devem ser escritos em italico,
%a primeira letra do primeiro nome deve ser em maiusculo e o restante em minusculo (inclusive a primeira letra do segundo nome).
% VEJA os exemplos abaixo.
%
% 1) voce nao quer que a secao fique com uppercase (caixa alta) automaticamente:
%\section[nouppercase]{\MakeUppercase{Estudo dos efeitos da radiacao ultravioleta C e TFD em celulas de} {\textit{Saccharomyces boulardii}}
%
% 2) por padrao os cases (maiusculas/minuscula) sao ajustados automaticamente, voce nao precisa usar makeuppercase e afins.
% \section{Introducao} % a introducao sera posta no texto como INTRODUCAO, automaticamente, como a norma indica.

\end{document} 